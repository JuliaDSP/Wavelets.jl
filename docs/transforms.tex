\documentclass[a4paper,11pt]{article}
\usepackage[english]{babel}
\usepackage[T1]{fontenc}
\usepackage{lmodern}
\usepackage{mathtools}
\usepackage{amssymb}

\title{Wavelets.jl Wavelet Transforms}
\author{Gudmundur Freyr Adalsteinsson} \date{\today}
\begin{document}

\maketitle

\section{Introduction} \label{sec:int}

This is a quick mathematical introduction to discrete wavelet transforms, and a technical specification of the implementation of the algorithms in the Wavelets.jl package.

\section{Discrete Orthogonal Wavelet Transforms}

Let the sequence $\{ \mathbf V_j \}_{j\in \mathbb Z}$ of subspaces of $\mathbf L^2(\mathbb R)$ be a multiresolution approximation. Let $\phi$ be a \emph{scaling function} wih (dyadic) dilations and translations denote by
\begin{equation} \label{eq:psi}
    \phi_{j,n}(t) = \frac{1}{\sqrt{2^j}} \phi\left(\frac{t-2^jn}{2^j}\right).
\end{equation}
With some condition on $\phi$ (details omitted here) the family $\{ \phi_{j,n} \}_{n\in \mathbb Z}$ is an orthonormal basis of $\mathbf V_j$. (Mallat)

The dilations of $\phi$ can be related to its translations,
\begin{equation}
    \frac{1}{\sqrt{2}} \phi(t/2) = \sum_{n = -\infty}^{\infty} h[n]\,\phi(t - n)
\end{equation}
with
\begin{equation}
    h[n] = \left\langle \frac{1}{\sqrt{2}} \phi(t/2), \phi(t - n) \right\rangle.
\end{equation}
The scaling function $\phi$ is completely specified by the discrete \emph{conjugate mirror filter} $h$ defined above. A necessary and sufficient condition on $h$ is that $|\hat h(\omega)|^2 + |\hat h(\omega + \pi)|^2 = 2$ and $\hat h(0) = \sqrt{2}$. A discrete approximation, $a_j$, of a function $f$ can be obtained from the expansion coefficents of the orthogonal projection of $f$ over $\mathbf V_j$
\begin{equation}
    a_j[n] =  \langle f, \phi_{j,n} \rangle
\end{equation}
and
\begin{equation}
    P_{\mathbf V_j}f = \sum_{n = -\infty}^{\infty} a_j[n]\,\phi_{j,n}
\end{equation}
is an approximation of $f$ at scale $2^j$.

Let $\mathbf W_j$ be the orthogonal complement of $\mathbf V_j$ in  $\mathbf V_{j-1}$. We can then decompose the projection over $\mathbf V_{j-1}f$
\begin{equation}
    P_{\mathbf V_{j-1}}f = P_{\mathbf V_j}f + P_{\mathbf W_j}f.
\end{equation}
A \emph{wavelet} $\psi$ is a function where
\begin{equation}
    \hat \psi(\omega) = \frac{1}{\sqrt 2} \hat g (\omega/2)\hat \phi (\omega/2)
\end{equation}
with
\begin{equation}
    \hat g(\omega) = e^{-i\omega} \hat h^{*}(\omega + \pi).
\end{equation}
Defining the dilations and translations of $\psi$ similarly to \eqref{eq:psi}
\begin{equation}
    \psi_{j,n}(t) = \frac{1}{\sqrt{2^j}} \psi\left(\frac{t-2^jn}{2^j}\right).
\end{equation}
The family $\{ \psi_{j,n} \}_{n\in \mathbb Z}$ is an orthonormal basis of $\mathbf W_j$ and $\{ \psi_{j,n} \}_{(j,n)\in \mathbb Z^2}$ is an orthonormal basis of $\mathbf L^2(\mathbb R)$. The mirror filter $g$ is explicitly defined as
\begin{equation}
    g[n] = \left\langle \frac{1}{\sqrt{2}} \psi(t/2), \phi(t - n) \right\rangle,
\end{equation}
the coefficients of
\begin{equation}
    \frac{1}{\sqrt{2}} \psi(t/2) = \sum_{n = -\infty}^{\infty} g[n]\,\phi(t - n).
\end{equation}
The filter $g$ is related to the filter $h$ by
\begin{equation}
    g[n] = (-1)^{1-n} h[1-n]
\end{equation}
and $h$ and $g$ can be viewed as low-pass and high-pass filters, respectively.

We can now decompose a function $f \in \mathbf L^2(\mathbb R)$ as
\begin{equation}
    f = \sum_{j = -\infty}^{\infty}\sum_{n = -\infty}^{\infty} \langle f, \psi_{j,n} \rangle\, \psi_{j,n}.
\end{equation}
From the properties of $\{\mathbf V_j\}_j$ and $\{\mathbf W_j\}_j$ we can also write the decomposition in terms of both $\phi$ and $\psi$, for any $J$
\begin{equation}
    f =
    \sum_{n = -\infty}^{\infty} a_J[n]\, \phi_{J,n} +
    \sum_{j = -\infty}^{J}\sum_{n = -\infty}^{\infty} d_j[n]\, \psi_{j,n}
\end{equation}
where $a_j[n] =  \langle f, \phi_{j,n} \rangle$ and $d_j[n] =  \langle f, \psi_{j,n} \rangle$. When $f$ is sufficiently smooth that $f\in \mathbf V_L$ where $L>J$, then
\begin{equation}
    f =
    \sum_{n = -\infty}^{\infty} a_J[n]\, \phi_{J,n} +
    \sum_{j = L - 1}^{J}\sum_{n = -\infty}^{\infty} d_j[n]\, \psi_{j,n}
\end{equation}
%===================================================
\end{document}
